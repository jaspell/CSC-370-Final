
% The \section{} command formats and sets the title of this
% section. We'll deal with labels later.
\section{Introduction}
\label{sec:intro}
Image processing is a vital component of computer vision and
artificial intelligence. Within image processing, image segmentation
continues to be a focus of research as machines are expected to be
able to have similar visual abilities as humans. Image segmentation
divides an image into regions of similar characteristics, which
ultimately can lead to object recognition. 

Depth images contain information about the distance a point in space
is to the camera eye. Whereas traditional RGB images contain a matrix
of pixels, each containing a red, green, and blue value, a depth image
only contains a matrix of depth values. Depth images can be visually
rendered as black and white images, where a point of maximum distance
is represented as white and a point of minimum distance is represented
as black. Accurate depth images were hard to construct until the
invention of modern technologies. Notably, the Xbox Kinect sensor
has the ability to simultaneously record depth images and color
images, providing researchers easy access to rich depth data.\\

There are two main approaches for segmenting an image: edge-based
segmentation and region-based segmentation \cite{aima}. Edge-based
segmentation locates major discontinuities in the image, indicating
separate distinguishable objects, whereas region-based segmentation
identifies different surfaces within those objects by calculating the
normal vectors at each point. The remainder of this paper will go over
background information pertaining to the two methods of segmentation,
describe our specific experimental design, and share our results. 

% In this section, you should introduce the reader to the problem you
% are attempting to solve. For example, for the first project: describe
% the $15$-puzzle, and why it's interesting as an A.I. problem. You
% should also cite and briefly describe other related papers that have
% tackled this problem in the past --- things that came up during the
% course of your research. In the AAAI style, citations look like
% \cite{aima} (see the comments in the source file \texttt{intro.tex} to
% see how this citation was produced). Conclude by summarizing how the
% remainder of the paper is organized. \\

% Citations: As you can see above, you create a citation by using the
% \cite{} command. Inside the braces, you provide a "key" that is
% uniue to the paper/book/resource you are citing. How do you
% associate a key with a specific paper? You do so in a separate bib
% file --- for this document, the bib file is called
% project1.bib. Open that file to continue reading...

% Note that merely hitting the "return" key will not start a new line
% in LaTeX. To break a line, you need to end it with \\. To begin a 
% new paragraph, end a line with \\, leave a blank
% line, and then start the next line (like in this example).
Overall, the aim in this section is context-setting: what is the
big-picture surrounding the problem you are tackling here?

