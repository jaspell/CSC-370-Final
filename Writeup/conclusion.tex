
\section{Conclusions}
\label{sec:concl}

We have developed two algorithms for depth image segmentation, one using the Marr-Hildreth algorithm (Laplacian Edge Detection) and one grouping pixels by normal vector (Gradient Surface Detection).  Those algorithms were tested on images of normal domestic scenes taken by an Xbox Kinect sensor, providing a realistic and varied environment.  We found that LED segmented based on jump discontinuities, locating disconnected objects, and GSD located surfaces within each object.

Because LED and GSD segment an image based on different qualities, their results provide different information.  LED finds disconnected objects, but provides no information about the shape of the object.  GSD describes the structure of each surface within the image, but cannot group surfaces to approximate a total object.  Further research in these algorithms could investigate a combination of the two, using LED to locate objects and GSD to separate them into surfaces, allowing an object to be recognized as a system of planes and facilitating automated modeling.  

We frequently found that characteristics in parts of an image -- small details, flaws in image quality, etc. -- governed the choice of parameters for the entire image, which limited the performance of both algorithms across the image as a whole.  An analysis function designed to tune parameters for sections of an image would allow the algorithms to perform effectively across the image, and would significantly increase the value of the algorithms presented.